% xcolor=table is because it seems that beamer uses the xcolor package in a
% strange way and thus don't accept us giving arguments to the package.
\documentclass[slidestop,compress,mathserif, xcolor=table]{beamer}

\usepackage[T1]{fontenc}
\usepackage[utf8]{inputenc}
\usepackage[english]{babel}


% \usepackage{mdwtab}
% \usepackage{mathenv}
% \usepackage{amsfonts}
% \usepackage{amsmath}
% \usepackage{amssymb}
% \usepackage{amsthm}

\usepackage{semantic}
\renewcommand{\ttdefault}{txtt} % Bedre typewriter font

% Use the NAT theme in uk (also possible in DK)
\usetheme[nat,uk, footstyle=low]{Frederiksberg}

% Make overlay sweet nice by having different transparancy depending on how
% "far" ahead the overlay is. AWSOME!!
\setbeamercovered{highly dynamic}
% possible to shift back, so they are just invisible untill they should overlay
% \setbeamercovered{invisible}

% Extend figures into either left or right margin
% Ex: \begin{narrow}{-1in}{0in} .. \end{narrow} will place 1in into left margin
\newenvironment{narrow}[2]{%
  \begin{list}{}{%
  \setlength{\topsep}{0pt}%
  \setlength{\leftmargin}{#1}%
  \setlength{\rightmargin}{#2}%
  \setlength{\listparindent}{\parindent}%
  \setlength{\itemindent}{\parindent}%
  \setlength{\parsep}{\parskip}}%
\item[]}{\end{list}}



\usepackage{tikz}
\usetikzlibrary{calc,shapes,arrows}
% below is for use of backgrounds (foregrounds are not used so they are not
% added to this specification).
\pgfdeclarelayer{background}
\pgfsetlayers{background,main}


\usepackage{subfigure}


% Write a short text to have that shown in the footer of slides other than the
% title slide.
\title[]{Regular-expression based bit coding}
% A possible subslide.
\subtitle{\tiny{TiPL - Topics in Programming Languages}}


\author{Jesper Reenberg}

% Only write DIKU in the footer of slides (except the title slide).
\institute[DIKU]{Department of Computer Science}

% Remove the date stamp from the footer of slides (except title slide) by giving
% it no short "text"
\date[]{\today} 

\begin{document}

\frame[plain]{\titlepage}


\begin{frame}
  \frametitle{Overview}

  What is covered

  \begin{itemize}
  \item What is actually implemented.

  \item Automatic XML schema generation

  \item Stuff that didn't make it to the report.

  \item Some miss explanation in the report.
  \end{itemize}
\end{frame}


\begin{frame}[c]
  \frametitle{What have we done -- 1/2}

  In summary we have implemented

  \begin{itemize}
  \item Backtracking parser for $Reg^\mu$ (inspired by Fritz's)

  \item DTD schema parser that generates $Reg$.
    \begin{itemize}
    \item Using HaXML for the DTD parsing.
    \end{itemize}

  \item Conversion of $Reg$ to $Reg^\mu$
    \begin{itemize}
    \item Replacing $\alpha^\ast$ with $\mu X . 1 + \alpha \times X$
    \end{itemize}

  \item Normalisation of $Reg^\mu$ (putting $X$ inside sums)

  \item Balancing of sum's in $Reg^\mu$.

  \item An optional specialisation of the $Reg^\mu$ w.r.t the desired data.

  \end{itemize}
\end{frame}

\begin{frame}[c]
  \frametitle{What have we done -- 2/2}
  \begin{itemize}

  \item XML parser that splits an XML file into its data and markup.
    \begin{itemize}
      
    \item Using HXT for the XML parsing.
    \item Specialised version of the DTD schema parser for the split markup
      file.
    \end{itemize}
  \item As seen from the experimental results we encoded a 1MB sample of the
    DBLP XML file.
    
    
  \item It was done from an automatically generated $Reg$ from the DBLP DTD schema.

  \item We actually also decoded it, and ended up with the
    \emph{same string as encoded}.
    
    \begin{itemize}
    \item So we assume it is ``correct''.
    \end{itemize}

  \end{itemize}
\end{frame}

\begin{frame}[c]
  \frametitle{Normal use of our implementation}
  \begin{itemize}

  \item Make a $Reg$ or automate one from a schema.

  \item Convert the $Reg$, Normalise and balance the $Reg^\mu$.

  \item Parse the desired data.

  \end{itemize}
\end{frame}

\begin{frame}[c]
  \frametitle{Automatic XML schema generation -- 1/2}
  
  We did investigate automatic schema generation from an XML file.

  \begin{itemize}
  \item XMLSpy ( \url{http://altova.com} commercial)
    \begin{itemize}
    \item Started paging and we aborted after quite some time.
    \end{itemize}

  \item relaxer (\url{http://relaxer.jp/})
    \begin{itemize}
    \item Failed to generate for a 200MB sample of the DBLP XML file. It
      used to much memory and JVM crashed even with JVM flags to enlarge the stack.
    \end{itemize}
  \end{itemize}

  The test machine had 2GB memory.

\end{frame}

\begin{frame}[c]
  \frametitle{Automatic XML schema generation -- 2/2}
  
  For smaller files they both generated considerably worse regular expressions
  than even the simplest (non-optimised) made by ourself.

  Drawbacks

  \begin{itemize}
  \item Can only generate a schema that matches the exact XML file. Not what the
    XML file may contain later. 
  \end{itemize}

  Conclusion 

  \begin{itemize}
  \item Not feasible for large files.

  \item Bad schema generation => bad regular expression generation => bad compression.
  \end{itemize}

\end{frame}

\begin{frame}[c]
  \frametitle{Further work}
  
  Additions to the further work mentioned in the report.

  \begin{itemize}
  \item Implementing coercion synthesis.

  \item Extend coercions with $Reg^\mu$.

  \end{itemize}

\end{frame}

\end{document}


\begin{frame}
  \frametitle{Clarification}
  
  The report may give an impression of some topics that is not true.

  \begin{itemize}
  \item We accept $Reg$ and transform them into $Reg^\mu$.

  \item Parser implemented on $Reg^\mu$ only.

  \item We only do specialisation (Huffman) on $Reg^\mu$ 
    \begin{itemize}
    \item not $Reg$ as is suggested in the report.
    \end{itemize}

  \end{itemize}
\end{frame}



%%% Local Variables: 
%%% mode: latex
%%% TeX-master: t
%%% End: 
