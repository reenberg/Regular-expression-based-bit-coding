% xcolor=table is because it seems that beamer uses the xcolor package in a
% strange way and thus don't accept us giving arguments to the package.
\documentclass[slidestop,compress,mathserif, xcolor=table]{beamer}


\usepackage[T1]{fontenc}
\usepackage[utf8]{inputenc}
\usepackage[english]{babel} 



% Use the NAT theme in uk (also possible in DK)
\usetheme[nat,uk, footstyle=low]{Frederiksberg}

% Make overlay sweet nice by having different transparancy depending on how
% "far" ahead the overlay is. AWSOME!!
\setbeamercovered{highly dynamic}
% possible to shift back, so they are just invisible untill they should overlay
%\setbeamercovered{invisible}


% Write a short text to have that shown in the footer of slides other than the
% title slide.
\title[]{TiPL - Topics in Programming Languages}
% A possible subslide.
\subtitle{Regular-expression based bit coding}


\author[Morten Brøns-Pedersen \and Jesper Reenberg \and Nis Wegmann]
       {Morten Brøns-Pedersen \and
        Jesper Reenberg  \\
        Nis Wegmann }



% Only write DIKU in the footer of slides (except the title slide).
\institute[DIKU]{Department of Computer Science}

% Remove the date stamp from the footer of slides (except title slide) by giving
% it no short "text"
\date[]{\today} 


% define nice colors for tables
\rowcolors[]{1}{green!20}{green!10}

\begin{document}

\frame[plain]{\titlepage}

\section{Bit coding}

\subsection{Introduction}

\begin{frame}
  \frametitle{Introduction}
  
  Ordinary texts ``waste'' space encoding information that is not used whereas
  regular expressions that have a smaller alphabet can use this as an advantage,
  and not encode information that can not be expressed within the given regular
  expression.

  With this idea we will show that it is possible to compress texts and get a
  worst case compression of 1 + overhead of representing the regular expression.

  Bit coding originates as a special case of Proof-Carrying Code.

\end{frame}


\begin{frame}
  \frametitle{Proof-Carrying Code -- (Ultra) Short intro}
  
  \begin{definition}[Weak]
    Is a technique that can verify properties about an application using formal
    proofs that accompanies the executable code.
  \end{definition}

  Originally described by George Necula and Peter Lee in 1996 (FIXME
  bibliography) to enforce system security policies such as

  \begin{itemize}
  \item Memory safety
  \item Buffer overflows
  \item Etc.
  \end{itemize}

  Specifically for use in kernel modules.

\end{frame}

\begin{frame}
  \frametitle{Proof-Carrying Code -- the special case}

  Originally Necula and Lee used formal proofs (fixme in some decution system)
  which had the disadvantage of being (fixme) 30\% larger than the executable
  code in the general case.

  In 2001 Necula and Rahul (FIXME bibliography) changed the fundamental idea of
  how PCC could be verified, introducing an oracle to guide the proof checker
  instead of the formal proof.

  This resulted in a drop down in ``proof'' size to (FIXME)
  
\end{frame}

\begin{frame}
  \frametitle{Which property to verify}
  
  So how does PCC relate to bit coding?

  If we choose membership testing as the property that needs to be verified,
  then we can look at the verification process as de-compression as it produces a
  wideness (the compressed text) and we can look at the oracle generation as the
  compression 
  
  Such a property could be membership testing
  
\end{frame}

\begin{frame}
  \frametitle{Abstract idea of bit compression using oracles}
  
  Show pictures/diagrams of the ``function'' mashines that we discussed, and
  thus lay down the ground idea of how a text and regex will result in a
  bitstream, and also how a regex and bitstream will result in the original text.

\end{frame}

\begin{frame}
  
  Thus it is important that the regular expression is constructed in a clever
  way, such that the representation of the regular expression is small
  
  (a|b|..|z)* vs [a-z]* 
  
  As discussed in previous talks about bit coding (and later) it is only the
  alternation and thus also the Kleene star ($1 + E \times E^*$) that generate
  any bits in the resulting bit code.  Thus it is important how the regular
  expression is expressed.

  \begin{block}{Simplified subset of \texttt{code} function}
\begin{semiverbatim}
  \center{code(inl v) => 0 \quad code(inr v) => 1}
\end{semiverbatim}
  \end{block}

  \begin{example}
    Text: dc\\
    Regex: a|(b|(c|d))
    
    \only<+>{( inr ( inr ( inr d ) ) , inr ( inr ( inl c ) ) )}
    \only<+>{\alert{( inr ( inr ( inr d ) ) , inr ( inr ( inl c ) ) )}}
    \only<+>{( \alert{inr ( inr ( inr d ) )} , inr ( inr ( inl c ) ) )}
    \only<+>{( 1 \alert{inr ( inr d )} , inr ( inr ( inl c ) ) )} 
    \only<+>{( 1 1 \alert{inr d} , inr ( inr ( inl c ) ) )} 
    \only<+>{( 1 1 1 , inr ( inr ( inl c ) ) )} 
    \only<+>{( 1 1 1 , \alert{inr ( inr ( inl c ) )} )}
    \only<+>{( 1 1 1 , 1 \alert{inr ( inl c )} )} 
    \only<+>{( 1 1 1 , 1 1 \alert{inl c} )} 
    \only<+>{( 1 1 1 , 1 1 0)} 
    \only<+>{1 1 1 1 1 0}
  \end{example}

\end{frame}

\begin{frame}

  \begin{block}{Simplified subset of \texttt{code} function}
\begin{semiverbatim}
  \center{code(inl v) => 0 \quad code(inr v) => 1}
\end{semiverbatim}
  \end{block}
  
  But if we shift the regular expression so d is in the beginning and then having 

  Where if we represent the regex as d|(a|(b|c)) we get

  \begin{example}
    Text: dc\\
    Regex: b|(c|(a|b))

    \only<+>{( inl d , inr ( inl c ) )}
    \only<+>{\alert{( inl d , inr ( inl c ) )}}
    \only<+>{( \alert{inl d} , inr ( inl c ) )}
    \only<+>{( 0 , inr ( inl c ) )}
    \only<+>{( 0 , \alert{inr ( inl c )} )}
    \only<+>{( 0 , 1 \alert{inl c} )}
    \only<+>{( 0 , 1 0 )}
    \only<+>{0 1 0}
  \end{example}

  (inl d, inr (inl c)) => 010.

  Thus statical analysis (minimising alphabet and occurrence checks) of the text
  to compress can sometime yield huge gains.
  In general this is not so easy, but structured texts as XML files, with a
  specifying schema it gets quite easy.

\end{frame}

\section{Bit coding}

\subsection{Theory}

\begin{frame}
  \frametitle{Naive usage}
  
\end{frame}

\begin{frame}
  \frametitle{Naive usage -- Practical example}
  Henglein Fig 8 med praktisk eksempel på hvilke bits der kommer ud
\end{frame}


\begin{frame}[c]
  \frametitle{Pop quiz}
  
  
  \begin{center}
    \huge{2min quiz}
  \end{center}

\end{frame}

\begin{frame}[c]
  \frametitle{Pop quiz -- Solutions}
  

  \begin{center}
    % temporarily disable the bad ass niceness highly dynamic fading of
    % overlays.
    \setbeamercovered{invisible}
    \begin{tabular}{l|l||c|c|c}
      \emph{Regex} & \emph{Text} & \emph{1} & \emph{x} & \emph{2} \\ \hline
      a* & aaaaa & 100000 & 000001 & \alert<2>{111110} \pause\pause \\
      (a|b)* & abaab & \alert<4>{10111010110} & 10111010111 & 01110101110 \pause\pause \\
      abdd & abdd & \alert<6>{10}     & 1 & 0 \pause\pause \\
      abdd & abdd & \alert<8>{10}     & 1 & 0 \vspace{1em} \pause\pause \\
      \emph{Regex} & \emph{Bits} & \emph{1} & \emph{x} & \emph{2} \\ \hline
      a* & 1110011 & \alert<10>{aaaa} & aaa & abab \pause\pause \\
      abdd & 0 & \alert<12>{abdd}     & $\epsilon$ & ddba \pause\pause \\
      abdd & 1 & abab     & \alert<14>{ddba} & $\epsilon$ \pause\pause \\
      abdd & 1 & abab     & ddba & \alert<16>{$\epsilon$} \pause\pause \\
    \end{tabular}
    
    % re-enable bad ass niceness hifly dynamic fading of overlays.
    \setbeamercovered{highly dynamic}
  \end{center}
  
\end{frame}

\subsection{Implementation}

\begin{frame}
  \frametitle{Implementation details}
  
  Frisch \& Cardelli dynamic backtracking algorithm

\end{frame}

\begin{frame}
  \frametitle{The naked code}
  
\end{frame}

\section{Preliminary results}

\begin{frame}
  \begin{center}
    \huge{live DEMO of code}
  \end{center}
\end{frame}

\begin{frame}
  \frametitle{Examined data}
  \begin{itemize}
  \item XML files
    
    \begin{itemize}
    \item DNA sequence
      
    \item DBLP database
    \end{itemize}

  \item RFC papers (technical english texts).
    
  \end{itemize}
\end{frame}

\begin{frame}
  \frametitle{Oracle vs gzip}
  
  off the shelf unix gzip
\end{frame}

\begin{frame}
  \frametitle{Oracle vs Hoffman}
  Compare text with small and large amount of entropy.
\end{frame}

\begin{frame}
  \frametitle{Disadvantages}

  \begin{itemize}
  \item Naive implementation yields(BIBTEX HENGLEIN) compression above 1
  \end{itemize}


\end{frame}

\begin{frame}
  \frametitle{Advantages}

  We have an idea of how to ensure compression below 1, which
\end{frame}









% \begin{frame} 
%   \frametitle{New style -- Oracle based guidance}

%   Pros.
%      Low memory usage, only a few bits needs to be read from the oracle at a time.
%   Cons.
%      Slower checking time.
     

%   Nondeterministic checker, but with a lot of tricks it is only a small part
%   that is nondeterministic and needs the oracles help.

%   Untrusted oracles is not a problem.

  
  
% \end{frame}

\end{document}





%%% Local Variables: 
%%% mode: latex
%%% TeX-master: t
%%% End: 
