% xcolor=table is because it seems that beamer uses the xcolor package in a
% strange way and thus don't accept us giving arguments to the package.
\documentclass[slidestop,compress,mathserif, xcolor=table]{beamer}

\usepackage[T1]{fontenc}
\usepackage[utf8]{inputenc}
\usepackage[english]{babel}


% \usepackage{mdwtab}
% \usepackage{mathenv}
% \usepackage{amsfonts}
% \usepackage{amsmath}
% \usepackage{amssymb}
% \usepackage{amsthm}

\usepackage{semantic}
\renewcommand{\ttdefault}{txtt} % Bedre typewriter font

% Use the NAT theme in uk (also possible in DK)
\usetheme[nat,uk, footstyle=low]{Frederiksberg}

% Make overlay sweet nice by having different transparancy depending on how
% "far" ahead the overlay is. AWSOME!!
\setbeamercovered{highly dynamic}
% possible to shift back, so they are just invisible untill they should overlay
% \setbeamercovered{invisible}

% Extend figures into either left or right margin
% Ex: \begin{narrow}{-1in}{0in} .. \end{narrow} will place 1in into left margin
\newenvironment{narrow}[2]{%
  \begin{list}{}{%
  \setlength{\topsep}{0pt}%
  \setlength{\leftmargin}{#1}%
  \setlength{\rightmargin}{#2}%
  \setlength{\listparindent}{\parindent}%
  \setlength{\itemindent}{\parindent}%
  \setlength{\parsep}{\parskip}}%
\item[]}{\end{list}}



\usepackage{tikz}
\usetikzlibrary{calc,shapes,arrows}
% below is for use of backgrounds (foregrounds are not used so they are not
% added to this specification).
\pgfdeclarelayer{background}
\pgfsetlayers{background,main}


\usepackage{subfigure}


% Write a short text to have that shown in the footer of slides other than the
% title slide.
\title[]{TiPL - Topics in Programming Languages}
% A possible subslide.
\subtitle{Regular-expression based bit coding}


\author[Morten Brøns-Pedersen \and Jesper Reenberg \and Nis Wegmann]
       {Morten Brøns-Pedersen \and
        Jesper Reenberg  \\
        Nis Wegmann }

% Only write DIKU in the footer of slides (except the title slide).
\institute[DIKU]{Department of Computer Science}

% Remove the date stamp from the footer of slides (except title slide) by giving
% it no short "text"
\date[]{\today} 

\begin{document}

\frame[plain]{\titlepage}

\section{Bit coding}

\subsection{Introduction}

\begin{frame}[c]
  \frametitle{Introduction}
     
  Since most regex represent smaller languages than the language of all strings
  ($\Sigma^\ast$), we expect to be able to encode strings belonging to such
  languages with less information than encoding with -- say -- latin1 or utf8. \newline

  \begin{block}{}
      Bit coding can bee seen as a special case of Proof-Carrying Code.
  \end{block}
  
\end{frame}

\begin{frame}
  \frametitle{Proof-Carrying Code -- (Ultra) Short intro 1/2}
  
  \begin{definition}[Weak]
    Is a technique that can verify properties about an application using formal
    proofs that accompanies the executable code.
  \end{definition}

  Originally described by George Necula and Peter Lee\cite{nele1996} to
  enforce system security policies such as

  \begin{itemize}
  \item Memory safety
  \item Buffer overflows
  \item Etc.
  \end{itemize}

  Specifically for use in kernel modules.

\end{frame}

\begin{frame}
  \frametitle{Proof-Carrying Code -- (Ultra) Short intro 2/2}

%   Necula and Lee used formal proofs (fixme in some decution system) which had
%   the disadvantage of being (fixme) 30\% larger than the executable code in the
%   general case.

%   Later Necula and Rahul\cite{nera2001} changed the fundamental idea of how
%   PCC could be verified, introducing an oracle to guide the proof checker
%   instead of the formal proof.

  \begin{itemize}
  \item Necula and Lee
    \begin{itemize}
    \item Disadvantages: formal proof representation was $\sim$1.000$\times$code size.
    \item Later \structure{$\sim$2.5$\times$code size}.
    \end{itemize}

  \item Necula and Rahul \cite{nera2001}
    \begin{itemize}
    \item Verification idea changed to use oracles as guides for proof checking.
    \item Dropped to \structure{12\% of code size}.
    \end{itemize}
  \end{itemize}
  
  \begin{block}{What is an oracle?}
    When one has to make a choice (e.g. when \structure{nondeterminism} arises) one can
    consult an oracle which \structure{will make the ``right'' choice} as if by
    magic.
  \end{block}

  
\end{frame}

\begin{frame}
  \frametitle{Proof-Carrying Code -- Which property to verify}
  
  So how does PCC relate to bit coding?\newline

  % If we choose membership testing as the property that needs to be verified,
%   then we can look at the verification process as de-compression as it produces a
%   witeness (the compressed text) and we can look at the oracle generation as the
%   compression 

  If

  \begin{block}{}
    \center{Membership testing is the property to be verified.}
  \end{block}

  then
  
  \begin{itemize}
    
  \item Encoding
    \begin{itemize}
    \item Oracle generation -- \structure{bits} -- that will guide the membership proof.
    \end{itemize}
    
  \item Decoding
    \begin{itemize}
    \item Produces a membership proof -- \structure{the original text} -- that
      belongs to the regular expression.
    \end{itemize}
    
  \end{itemize}
  
\end{frame}

\begin{frame}
  \frametitle{Review of Henglein \& Nielsens\cite{heni2010} axiomatization.}

  4 main results

  \begin{itemize}
  \item \structure{Regular expressions as types.} A regular expression is a set of proofs of
    membership rather than strings.
  \item \structure{Computational interpretation of containment.} E.g a total function from
    proofs in a regular expressions to proofs in a containing one (coercions).
  \item \structure{Parametric completeness} ($|- c : E[X_1, \ldots, X_n] \leq F[X_1,
    \ldots, X_n]$).
  \item \structure{Oracle based bit coding} of proof values and transformation of oracles.
  \end{itemize}
\end{frame}

\begin{frame}
  \frametitle{Recall membership and inhabitation:}

  

  \begin{figure}[!h]
    
    \begin{columns}[totalwidth=0.8\textwidth]
      \subfigure[Membership proofs]{        
        \begin{column}{.5\textwidth}
          \[
          \inference{}{\epsilon \in 1}
          \]
          \[
          \inference{}{\texttt{a} \in \texttt{a}}
          \]
          \[
          \inference{s \in E}{s \in E + F}
          \]
          \[
          \inference{s \in F}{s \in E + F}
          \]
          \[
          \inference{s \in E & t \in F}{st \in E \times F}
          \]
          \[
          \inference{s \in 1 + E \times E^{\ast}}{s \in E^{\ast}}
          \] \\[1em]  
        \end{column}
      }
      \subfigure[Inhabitation proofs]{
        \begin{column}{.5\textwidth}
          \[
          \inference{}{() : 1}
          \]
          \[
          \inference{}{\texttt{a} : \texttt{a}}
          \]
          \[
          \inference{v : E}{\mathtt{inl}\ v : E + F}
          \]
          \[
          \inference{v : F}{\mathtt{inr}\ v : E + F}
          \]
          \[
          \inference{v : E & w : F}{(v, w) : E \times F}
          \]
          \[
          \inference{v : 1 + E \times E^{\ast}}{\mathtt{fold}\ v : E^{\ast}}
          \]\\[1em]  
        \end{column}
      }
    \end{columns}
    
    \caption{Regular expression membership and inhabitation}
  \end{figure}

\end{frame}

\begin{frame}
  \frametitle{Using an oracle.}
  So lets try and construct a proof of membership in $\mathcal{L}|[\mathtt{a}
  \times (\mathtt{b} \times \mathtt{c})^{\ast}|]$ for string \textit{abc}.\\[1em]
  
  \setbeamercovered{invisible}
  \uncover<2-18>{
    \indent Proof:
    {\tiny
      \[
      \inference{
        \uncover<3->{
          \inference{
          }{
            \mathtt{a} : \mathtt{a}
          }
        } &
        \uncover<5->{
          \inference{
            \uncover<6->{
              \inference{
                \uncover<8->{
                  \inference{
                    \uncover<9->{
                      \inference{
                        \uncover<10->{
                          \inference{
                          }{
                            \mathtt{b} : \mathtt{b}
                          }
                        } &
                        \uncover<12->{
                          \inference{
                          }{
                            \mathtt{c} : \mathtt{c}
                          }
                        }
                      }{
                        (
                        \uncover<10->{
                          \mathtt{b}
                        },
                        \uncover<12->{
                          \mathtt{c}
                        }
                        ) : \alert<9-10>{\mathtt{b}} \times \alert<11-12>{\mathtt{c}}
                      }
                    } &
                    \uncover<14->{
                      \inference{
                        \uncover<15->{
                          \inference{
                            \uncover<17->{
                              \inference{
                              }{
                                () : 1
                              }
                            }
                          }{
                            \alt<15>{\alert<15>{\mathtt{inl}\ \text{or}\ \mathtt{inr}\text{?}}}
                            {\mathtt{inl} \uncover<17->{()}\hspace{17pt}} : \alert<16-17>{1} + (\mathtt{b} \times \mathtt{c}) \times
                            (\mathtt{b} \times \mathtt{c})^{\ast}
                          }
                        }
                      }{
                        \mathtt{fold}\ \uncover<16->{\mathtt{inl} \uncover<17->{()}} : (\mathtt{b} \times \mathtt{c})^{\ast}
                      }
                    }
                  }{
                    (
                    \uncover<9->{
                      (
                      \uncover<10->{
                        \mathtt{b}
                      },
                      \uncover<12->{
                        \mathtt{c}
                      }
                      )
                    },
                    \uncover<14->{
                      \mathtt{fold}\ \uncover<16->{\mathtt{inl} \uncover<17->{()}}
                    }
                    ) :
                    \alert<6-12>{(\mathtt{b} \times \mathtt{c})} \times
                    \alert<13-17>{(\mathtt{b} \times \mathtt{c})^{\ast}}
                  }
                }
              }{
                \alt<6>{\alert<6>{\mathtt{inl}\ \text{or}\ \mathtt{inr}\text{?}}}
                {
                  \mathtt{inr} (
                  \uncover<8->{
                    (
                    \uncover<9->{
                      (
                      \uncover<10->{
                        \mathtt{b}
                      },
                      \uncover<12->{
                        \mathtt{c}
                      }
                      )
                    },
                    \uncover<14->{
                      \mathtt{fold}\ \uncover<16->{\mathtt{inl} \uncover<17->{()}}
                    }
                    )
                  }
                  )
                }
                : 1 + \alert<7-17>{(\mathtt{b} \times \mathtt{c}) \times (\mathtt{b} \times \mathtt{c})^{\ast}}
              }
            }
          }{
            \mathtt{fold}(
            \uncover<7->{
              \mathtt{inr} (
              \uncover<8->{
                (
                \uncover<9->{
                  (
                  \uncover<10->{
                    \mathtt{b}
                  },
                  \uncover<12->{
                    \mathtt{c}
                  }
                  )
                },
                \uncover<14->{
                  \mathtt{fold}\ \uncover<16->{\mathtt{inl} \uncover<17->{()}}
                }
                )
              }
              )
            }
            ) : (\mathtt{b} \times \mathtt{c})^{\ast}
          }
        }
      }{
        (\uncover<3->{\mathtt{a}},
        \uncover<5->{\mathtt{fold}(
          \uncover<7->{
            \mathtt{inr} (
            \uncover<8->{
              (
              \uncover<9->{
                (
                \uncover<10->{
                  \mathtt{b}
                },
                \uncover<12->{
                  \mathtt{c}
                }
                )
              },
              \uncover<14->{
                \mathtt{fold}\ \uncover<16->{\mathtt{inl} \uncover<17->{()}}
              }
              )
            }
            )
          }
          )
        }
        ) : \alert<2-3>{\mathtt{a}} \times \alert<4-17>{(\mathtt{b} \times \mathtt{c})^{\ast}}
      }
      \]
    }
  Oracle: \uncover<7->{1}\uncover<16->{0}
  }
  \\[2em]
  
  \uncover<18>{\structure{Recall findings by Necula: proof proof size vs Oracle size}}

  \setbeamercovered{highly dynamic}
\end{frame}

\begin{frame}
  \frametitle{Nondeterminism}
  There is only one place where the next rule in the proof cannot be determined
  by the structure of the regular expression. Namely alternation.

  \begin{center}
    \structure{\[ \inference{v : E}{\mathtt{inl}\ v : E + F}
      \]
      \[
      \inference{v : E}{\mathtt{inr}\ v : E + F}
      \]}
  \end{center}

    Either choose \texttt{inl} or \texttt{inr}.

\end{frame}

\begin{frame}
  \frametitle{Nondeterministic membership prover -- Abstract view}

  \tikzstyle{block} = [draw, fill=blue!20, rectangle, minimum height=3em,
  minimum width=6em]
  
  \tikzstyle{fun} = []

  \tikzstyle{machine} = [fill=blue!20, rectangle, rounded corners,
  minimum width = 4em, minimum height = 3em]
  
  \begin{figure}[ht]
    \centering
    
    \subfigure[Encoding]{
      \begin{tikzpicture}
        \node [machine]  (machine) {encode};
        
        \node [fun,left of=machine, xshift=-0.7cm]  (text) {Text};
        \node [fun, above of=machine, yshift=0.2cm] (regex) {Regular expression};
        
        \node [fun, right of=machine, xshift=0.7cm] (bits) {Bits};
        
        \draw[->] (regex) -- (machine);
        \draw[->] (text) -- (machine);
        \draw[->] (machine) -- (bits);
        
        % print a nice background
        \begin{pgfonlayer}{background}
          \path[fill=green!10,rounded corners]
          ($(current bounding box.south west) - (.2,.2)$) rectangle
          ($(current bounding box.north east) + (.2,.2)$);
        \end{pgfonlayer}    
      \end{tikzpicture}
    }

    \subfigure[Decode]{
      \begin{tikzpicture}
        \node [machine]  (machine) {decode};
        
        \node [fun,left of=machine, xshift=-0.7cm]  (text) {Text};
        \node [fun, above of=machine, yshift=0.2cm] (regex) {Regular expression};
        
        \node [fun, right of=machine, xshift=0.7cm] (bits) {Bits};
        
        \draw[->] (regex) -- (machine);
        \draw[<-] (text) -- (machine);
        \draw[<-] (machine) -- (bits);
        
        % print a nice background
        \begin{pgfonlayer}{background}
          \path[fill=green!10,rounded corners]
          ($(current bounding box.south west) - (.2,.2)$) rectangle
          ($(current bounding box.north east) + (.2,.2)$);
        \end{pgfonlayer}    
        
      \end{tikzpicture}
    }

  \end{figure}
  
\end{frame}

% \begin{frame}
%   \frametitle{Constructing an oracle.}
%   Lets construct an oracle for the proof of membership of the string
%   \texttt{foobarbaz} in $\mathcal{L}|[E|]$.\\[1em]
%   \only<+>{
%     $.$
%   }
%   \only<+>{
%     $\phantom{.}.$
%   }
%   \only<+>{
%     $\phantom{..}.$
%   }
%   \only<+>{
%     $\phantom{...}.$
%   }
%   \only<+>{
%     $\phantom{....}.$
%   }
% \end{frame}

\begin{frame}
  \frametitle{Oracle based bit coding.}

  \only<+>{
    Now we've seen how to use and create oracles. Here are the rules:
    \begin{block}{Encoding}
      \begin{eqnarray*}
        code(() : 1) &=& \epsilon \\
        code(\mathtt{a} : \mathtt{a}) &=& \epsilon \\
        code(\mathtt{inl}\ v : E + F) &=& \mathtt{0}code(v : E) \\
        code(\mathtt{inr}\ v : E + F) &=& \mathtt{1}code(v : F) \\
        code((v, e) : E \times F)     &=& code(v : E)code(w : F) \\
        code(\mathtt{fold}\ v : E^{\ast}) &=& code(v : 1 + E \times E^{\ast})
      \end{eqnarray*}
    \end{block}
  }
  \only<+>{
    \begin{block}{Decoding}
      {\scriptsize
      \begin{eqnarray*}
        decode'(d : 1) &=& (() : 1) \\
        decode'(d : \mathtt{a}) &=& (\mathtt{a}, d) \\
        decode'(\mathtt{0}d : E + F) &=& \mathbf{let}\ (v, d') <- decode'(d : E) \\
        && \mathbf{in}\ (\mathtt{inl}\ v, d') \\
        decode'(\mathtt{1}d : E + F) &=& \mathbf{let}\ (v, d') <- decode'(d : F) \\
        && \mathbf{in}\ (\mathtt{inr}\ v, d') \\
        decode'(d : E \times F) &=& \mathbf{let}\ (v, d') <- decode'(d : E) \\
        && \phantom{\mathbf{let}}\ (w, d'') <- decode'(d' : F) \\
        && \mathbf{in}\ ((v, w), d'') \\
        decode'(d : E^{\ast}) &=& \mathbf{let}\ (v, d') <- decode'(d : 1 + E
        \times E^{\ast}) \\
        && \mathbf{in}\ (\mathtt{fold}\ v, d') \\
        decode(d : E) &=& \mathbf{let}\ (v, d') <- decode'(d : E) \\
        && \mathbf{in}\ \mathbf{if}\ d' = \epsilon\ \mathbf{then}\ v\
        \mathbf{else}\ error
      \end{eqnarray*}
    }
    \end{block}
  }
\end{frame}

\begin{frame}
  \frametitle{Choosing the right regular expression -- 1/2}

  Oracles representing (proofs representing) the same string may vary in size
  with the choice of regular expression.\\[1em]

  Remember: \structure{\texttt{inl} is 0}, \structure{\texttt{inr} is 1}.

  \begin{example}
    Text: \texttt{dc}\\
    Regex: \texttt{a} + (\texttt{b} + (\texttt{c} + \texttt{d}))

    \only<+>{( \texttt{inr} ( \texttt{inr} ( \texttt{inr} \texttt{d} ) ) , \texttt{inr} ( \texttt{inr} ( \texttt{inl} \texttt{c} ) ) )}
    \only<+>{\alert{( \texttt{inr} ( \texttt{inr} ( \texttt{inr} \texttt{d} ) ) , \texttt{inr} ( \texttt{inr} ( \texttt{inl} \texttt{c} ) ) )}}
    \only<+>{( \alert{\texttt{inr} ( \texttt{inr} ( \texttt{inr} \texttt{d} ) )} , \texttt{inr} ( \texttt{inr} ( \texttt{inl} \texttt{c} ) ) )}
    \only<+>{( 1 \alert{\texttt{inr} ( \texttt{inr} \texttt{d} )} , \texttt{inr} ( \texttt{inr} ( \texttt{inl} \texttt{c} ) ) )} 
    \only<+>{( 1 1 \alert{\texttt{inr} \texttt{d}} , \texttt{inr} ( \texttt{inr} ( \texttt{inl} \texttt{c} ) ) )} 
    \only<+>{( 1 1 1 , \texttt{inr} ( \texttt{inr} ( \texttt{inl} \texttt{c} ) ) )} 
    \only<+>{( 1 1 1 , \alert{\texttt{inr} ( \texttt{inr} ( \texttt{inl} \texttt{c} ) )} )}
    \only<+>{( 1 1 1 , 1 \alert{\texttt{inr} ( \texttt{inl} \texttt{c} )} )} 
    \only<+>{( 1 1 1 , 1 1 \alert{\texttt{inl} \texttt{c}} )} 
    \only<+>{( 1 1 1 , 1 1 0)} 
    \only<+>{1 1 1 1 1 0}
  \end{example}

\end{frame}

\begin{frame}
  \frametitle{Choosing the right regular expression -- 2/2}

  Oracles representing (proofs representing) the same string may vary in size
  with the choice of regular expression.\\[1em]

  Remember: \structure{\texttt{inl} is 0}, \structure{\texttt{inr} is 1}.

  \begin{example}
    Text: \texttt{dc}\\
    Regex: \texttt{d} + (\texttt{c} + (\texttt{a} + \texttt{b}))

    \only<+>{( \texttt{inl} \texttt{d} , \texttt{inr} ( \texttt{inl} \texttt{c} ) )}
    \only<+>{\alert{( \texttt{inl} \texttt{d} , \texttt{inr} ( \texttt{inl} \texttt{c} ) )}}
    \only<+>{( \alert{\texttt{inl} \texttt{d}} , \texttt{inr} ( \texttt{inl} \texttt{c} ) )}
    \only<+>{( 0 , \texttt{inr} ( \texttt{inl} \texttt{c} ) )}
    \only<+>{( 0 , \alert{\texttt{inr} ( \texttt{inl} \texttt{c} )} )}
    \only<+>{( 0 , 1 \alert{\texttt{inl} \texttt{c}} )}
    \only<+>{( 0 , 1 0 )}
    \only<+>{0 1 0}
  \end{example}

  \only<8>{Statistical analysis of the data to encode, suggests equivalent regular
    expressions that yield higher compression rates.}
\end{frame}

% \begin{frame}
%   Maybe: Practical use:
%   [example (motivation for coercions): a piece of code where each branch of a
%   conditional returns oracles belonging to different regex'es, all contained in a
%   'master' regex OR regular expression adapting to network traffic]
% \end{frame}

\begin{frame}
  \frametitle{Remember coercions?}
  \only<+>{
    A coercion is a proof of containment...

    \begin{quote}
      [Figure 9.]
    \end{quote}
  }
  \only<+>{
    ...with a computational interpretation.
    {\tiny
      \begin{columns}[totalwidth=0.8\textwidth]
        \begin{column}{.5\textwidth}
          \begin{eqnarray*}
      retag(\mathtt{inl} v)                        &=& \mathtt{inr} v\\
      retag(\mathtt{inr} v)                        &=& \mathtt{inl} v\\
      retag^{-1}                                   &=& retag\\
      tagL (v)                                     &=& \mathtt{inl} v\\
      untag (\mathtt{inl} v)                       &=& v\\
      untag (\mathtt{inr} v)                       &=& v\\
      shuffle(\mathtt{inl} v)                      &=& \mathtt{inl} (\mathtt{inl} v)\\
      shuffle(\mathtt{inr} (\mathtt{inl} v))       &=& \mathtt{inl} (\mathtt{inr} v)\\
      shuffle(\mathtt{inr} (\mathtt{inr} v))       &=& \mathtt{inr} v\\
      shuffle^{-1} (\mathtt{inl} (\mathtt{inl} v)) &=& \mathtt{inl} v\\
      shuffle^{-1} (\mathtt{inl} (\mathtt{inr} v)) &=& \mathtt{inr} (\mathtt{inl} v)\\
      shuffle^{-1} (\mathtt{inr} v)                &=& \mathtt{inr} (\mathtt{inr} v)\\
      swap(v, w)                                   &=& (w, v)\\
      swap^{-1}                                    &=& swap\\
      proj(v, w)                                   &=& w\\
      proj^{-1} (w)                                &=& ((), w)\\
      assoc(v, (w, x))                             &=& ((v, w), x)
    \end{eqnarray*}
  \end{column}
  \begin{column}{.5\textwidth}
          \begin{eqnarray*}
      assoc^{-1} ((v, w), x)                       &=& (v, (w, x))\\
      distL(v, \mathtt{inl} w)                     &=& \mathtt{inl} (v, w)\\
      distL(v, \mathtt{inr} x)                     &=& \mathtt{inr} (v, x)\\
      distL^{-1} (\mathtt{inl} (v, w))             &=& (v, \mathtt{inl} w)\\
      distL^{-1} (\mathtt{inr} (v, x))             &=& (v, \mathtt{inr} x)\\
      distR(\mathtt{inl} v, w)                     &=& \mathtt{inl} (v, w)\\
      distR(\mathtt{inr} v, x)                     &=& \mathtt{inr} (v, x)\\
      distR^{-1} (\mathtt{inl} (v, w))             &=& (\mathtt{inl} v, w)\\
      distR^{-1} (\mathtt{inr} (v, x))             &=& (\mathtt{inr} v, x)\\
      fold (v)                                     &=& \mathtt{fold} v\\
      fold^{-1} (\mathtt{fold} v)                  &=& v\\
      (c + d)(inl v)                                      &=& \mathtt{inl} (c(v))\\
      (c + d)(inr w)                                      &=& \mathtt{inr} (d(w))\\
      (c \times d)(v, w)                                       &=& (c(v), d(w))\\
      (c; d)(v)                                           &=& d(c(v))\\
      id(v)                                        &=& v\\
      (fix f.c)(v)                                 &=& c[fix f.c / f](v)
    \end{eqnarray*}
  \end{column}
  \end{columns}
  }
  }
  \only<+>{
    \begin{block}{Definition 3 (Computational interpretation).}
      {\it
        The computational interpretation of (closed) coercion $c : E \leq F$ is the
        partial function $\mathcal{F}|[c|] : E -> F$ defined by $\mathcal{F}|[c|](v) = w$ if $c(v) = w$ can
        be derived by the equational theory in Figure 10. We say $c$ is total if
        $\mathcal{F}|[c|]$ is total.
      }
    \end{block}
    In short: Coercions transform our membership proofs from one regular
    expression to another.
  }
\end{frame}

\begin{frame}[c]
  \frametitle{Coercion synthesis}

  
  \begin{itemize}
  \item Coercions can be synthesised automatically.

  \item If $E \leq F$ then there exists a coercion $c : E \leq F$.

  \item If there exists a coercion then the coercion synthesis will find
    one. Else it produces a witness $s \in \mathcal{L}|[E|] \backslash
    \mathcal{L}|[F|]$.
  \end{itemize}


  % \begin{quote}
    % [Figure 19 for 'wow' effect.]
  % \end{quote}
\end{frame}

\begin{frame}
  \frametitle{Choosing the right regular expression (revisited)}

  \begin{quote}
    Let
    \begin{eqnarray*}
      E &=& \mathtt{a} + (\mathtt{b} + (\mathtt{c} + \mathtt{d})),\\
      F &=& \mathtt{d} + (\mathtt{c} + (\mathtt{a} + \mathtt{b})).
    \end{eqnarray*}
  \end{quote}
  For our particular membership proofs (strings) we expect the oracles for $F$
  to be smaller than for $E$.\\[1em]

  Clearly $\mathcal{L}|[E|] = \mathcal{L}|[F|]$ and thus $\mathcal{L}|[E|]
  \subseteq \mathcal{L}|[F|]$. By the soundness and completeness of the rules
  for containment we have $|- c : E \leq F$.\\[1em]

  So we can convert $v \in E$ to $w \in F$ by $\mathcal{F}|[c|]$ (and of course
  $||v|| = ||w||$).
\end{frame}

\begin{frame}
  \frametitle{Coercion of oracles}

  Find a $\hat{c}$ that converts oracles for proofs in $E$ to oracles for proofs
  in $E'$ (provided $E \leq E'$).

    \begin{itemize}
    \item Naïvely: If $|- c : E \leq E'$, then $\mathcal{F}|[\hat{c}|] =
      code \circ \mathcal{F}|[c|] \circ decode$.
    \item Directly:
      \begin{quote}
        [Figure 23.]
      \end{quote}
    \end{itemize}
\end{frame}
% END (From udkast_morten.txt)

\subsection{Pop quiz}

\begin{frame}[c]
  \frametitle{Pop quiz}
  
  
  \begin{center}
    \huge{2min quiz}
  \end{center}

\end{frame}

\begin{frame}[c]
  \frametitle{Pop quiz -- Solutions}
 
  % temporarily disable the bad ass niceness highly dynamic fading of
  % overlays.
  \setbeamercovered{invisible}
    
  \begin{narrow}{-3em}{0in}
    % define nice colors for tables
    \rowcolors[]{1}{green!20}{green!10}
    
    
    \footnotesize{
      \begin{tabular}{l|l||c|c|c}
        \emph{Regex} & \emph{Text} & \emph{1} & \emph{x} & \emph{2} \\ \hline
        $a^\ast$ & aaaaa & 100000 & 000001 & \alert<2>{111110} \pause\pause \\
        $(a + b)^\ast$ & abaab & \alert<4>{10111010110} & 10111010111 & 01110101110 \pause\pause \\
        $a \times b \times d \times d$ & abdd & 0 & 1 & \alert<6>{$\epsilon$} \pause\pause \\
        $a \times b^\ast \times a$ & abba & 11 & \alert<8>{110} & 10
        \vspace{1em} \pause\pause \\
        \emph{Regex} & \emph{Bits} & \emph{1} & \emph{x} & \emph{2} \\ \hline
        $((o + b) + g)^\ast$ & 100111010 & og & obg & \alert<10>{ogb} \pause\pause \\
        $(m + s)^\ast \times (t + a)^{\ast}$ & 1110011100 & \alert<12>{smat} &
        $\epsilon$ & ma \pause\pause \\
      \end{tabular}
    }
    % re-enable bad ass niceness hifly dynamic fading of overlays.

  \end{narrow}


  \begin{block}<1->{The hard one...}
    
    Regular expression: $(a \times ((b \times b) + c)^\ast) \times a$ \\
    Bits: 1011101111100\\
    Text: \pause abbcbbccbba
  \end{block}

    \setbeamercovered{highly dynamic}  
\end{frame}

\subsection{Implementation}

\begin{frame}[c]
  \frametitle{The naked code}

  \begin{center}
    \Huge{live DEMO}
  \end{center}
  
\end{frame}

\section{Preliminary results}

\begin{frame}[fragile]
  \frametitle{Preliminary results}

  Compression of a subset of the following xml-file from DBLP (the Digital
  Bibliography and Library Project)

{\tiny
\begin{verbatim}
<?xml version="1.0"?>
<dblptags>
<tag key="conf/rsctc/WengZ04">Plagiarized Papers in DBLP</tag>
<tag key="books/ph/KemperM94">Access Support Relations in Textbooks</tag>
<tag key="conf/icde/LitwinL86">1986</tag>
<tag key="journals/software/LitwinL87">1987</tag>
...
</dblptags>
\end{verbatim}
} 

  ... yields the following results:

  \begin{itemize}
  \item Before compression: 14032 bits
  \item Compression using regexps (unbalanced): 31120 bits
  \item Compression using regexps (balanced): 7340 bits
  \item Compression using gzip: 4352 bits
  \end{itemize}

\end{frame}

\begin{frame}[fragile]
  \frametitle{Future Work}

  In our project we would like to investigate these topics:

  \begin{itemize}
  \item How to automatically derive a regular expression from a string---Special
    case: given an xml-scheme derive an efficient regular expression.
  \item How to automatically rewrite a regular expression with respect to a
    given string to achieve a good compression ratio.
  \item Provide a new formalization/representation of regular expressions more
    suitable for data-compression.
  \item Find a criteria for strings for determining the theoretical maximum
    achievable compression ratio using regular expressions.
  \end{itemize}

\end{frame}

% \begin{frame} 
%   \frametitle{New style -- Oracle based guidance}

%   Pros.
%      Low memory usage, only a few bits needs to be read from the oracle at a time.
%   Cons.
%      Slower checking time.
     

%   Nondeterministic checker, but with a lot of tricks it is only a small part
%   that is nondeterministic and needs the oracles help.

%   Untrusted oracles is not a problem.

\begin{frame}
  \frametitle{Bibliography}
  
  \bibliographystyle{../bibliography/theseurl}
  \bibliography{../bibliography/bibliography}
  
\end{frame}

\end{document}





%%% Local Variables: 
%%% mode: latex
%%% TeX-master: t
%%% End: 
