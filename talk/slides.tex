% xcolor=table is because it seems that beamer uses the xcolor package in a
% strange way and thus don't accept us giving arguments to the package.
\documentclass[slidestop,compress,mathserif, xcolor=table]{beamer}


\usepackage[T1]{fontenc}
\usepackage[utf8]{inputenc}
\usepackage[english]{babel} 



% Use the NAT theme in uk (also possible in DK)
\usetheme[nat,uk, footstyle=low]{Frederiksberg}

% Make overlay sweet nice by having different transparancy depending on how
% "far" ahead the overlay is. AWSOME!!
\setbeamercovered{highly dynamic}
% possible to shift back, so they are just invisible untill they should overlay
%\setbeamercovered{invisible}


% Write a short text to have that shown in the footer of slides other than the
% title slide.
\title[]{TiPL - Topics in Programming Languages}
% A possible subslide.
\subtitle{Regular-expression based bit coding}


\author[Morten Brøns-Pedersen \and Jesper Reenberg \and Nis Wegmann]
       {Morten Brøns-Pedersen \and
        Jesper Reenberg  \\
        Nis Wegmann }



% Only write DIKU in the footer of slides (except the title slide).
\institute[DIKU]{Department of Computer Science}

% Remove the date stamp from the footer of slides (except title slide) by giving
% it no short "text"
\date[]{\today} 


% define nice colors for tables
\rowcolors[]{1}{green!20}{green!10}

\begin{document}

\frame[plain]{\titlepage}

\section{Bit coding}

\subsection{Introduction}

\begin{frame}
  \frametitle{Introduction}
  
  maybe sketch bit coding as described in precious talks.

\end{frame}

\begin{frame}
  \frametitle{Purpose}

  We have made a formal definition of what bit coding is when used to compress
  texts

  \begin{definition}
    Compress a text to .... Morten had a great def.
  \end{definition}
\end{frame}

\begin{frame}
  \frametitle{Disadvantages}
  
  As will show the naive way (as described in Henglein fix bibliography) sux as it yields compression above 1

\end{frame}

\begin{frame}
  \frametitle{Advantages}
  
  Luckily we will demonstrate ways of optimising and getting compression below 1
  
\end{frame}

\begin{frame}
  \frametitle{Representation}
  
  Whut? which representation?

\end{frame}


\section{PCC - Proof-Carrying Code}

\subsection{Introduction}

\begin{frame}
  \frametitle{(Ultra) Short intro}
  
  What is PCC and how can it be used in its 

\end{frame}

\begin{frame}
  \frametitle{Full proofs vs oracle}
  
\end{frame}

\begin{frame}
  \frametitle{Oracle rules...}

  Improvements of oracle vs proofs
  
\end{frame}


\begin{frame}
  \frametitle{Abstract idea of bit compression using oracles}
  
  Show pictures/diagrams of the ``function'' mashines that we discussed, and
  thus lay down the ground idea of how a text and regex will result in a
  bitstream, and also how a regex and bitstream will result in the original text.

\end{frame}

\section{Bit coding}

\subsection{Theory}

\begin{frame}
  \frametitle{Naive usage}
  
\end{frame}

\begin{frame}
  \frametitle{Naive usage -- Practical example}
  Henglein Fig 8 med praktisk eksempel på hvilke bits der kommer ud
\end{frame}


\begin{frame}[c]
  \frametitle{Pop quiz}
  
  
  \begin{center}
    \huge{2min quiz}
  \end{center}

\end{frame}

\begin{frame}[c]
  \frametitle{Pop quiz -- Solutions}
  

  \begin{center}
    % temporarily disable the bad ass niceness highly dynamic fading of
    % overlays.
    \setbeamercovered{invisible}
    \begin{tabular}{l|l||c|c|c}
      \emph{Regex} & \emph{Text} & \emph{1} & \emph{x} & \emph{2} \\ \hline
      a* & aaaaa & 100000 & 000001 & \alert<2>{111110} \pause\pause \\
      (a|b)* & abaab & \alert<4>{10111010110} & 10111010111 & 01110101110 \pause\pause \\
      abdd & abdd & \alert<6>{10}     & 1 & 0 \pause\pause \\
      abdd & abdd & \alert<8>{10}     & 1 & 0 \vspace{1em} \pause\pause \\
      \emph{Regex} & \emph{Bits} & \emph{1} & \emph{x} & \emph{2} \\ \hline
      a* & 1110011 & \alert<10>{aaaa} & aaa & abab \pause\pause \\
      abdd & 0 & \alert<12>{abdd}     & $\epsilon$ & ddba \pause\pause \\
      abdd & 1 & abab     & \alert<14>{ddba} & $\epsilon$ \pause\pause \\
      abdd & 1 & abab     & ddba & \alert<16>{$\epsilon$} \pause\pause \\
    \end{tabular}
    
    % re-enable bad ass niceness hifly dynamic fading of overlays.
    \setbeamercovered{highly dynamic}
  \end{center}
  
\end{frame}


\begin{frame}
  \frametitle{Diagram... make better name}
  
  Figurer over oracle bits og text som ``maskiner''
  Bør måske flyttes til introduktionen så man får en mere intuitiv ide om
  hvordan tingene relaterer til hinanden før man bliver ``overfaldet'' med teori.
  
\end{frame}


\subsection{Implementation}

\begin{frame}
  \frametitle{Implementation details}
  
  Frisch \& Cardelli dynamic backtracking algorithm

\end{frame}

\begin{frame}
  \frametitle{The naked code}
  
\end{frame}

\section{Preliminary results}

\begin{frame}
  \begin{center}
    \huge{live DEMO of code}
  \end{center}
\end{frame}

\begin{frame}
  \frametitle{Examined data}
  \begin{itemize}
  \item XML files
    
    \begin{itemize}
    \item DNA sequence
      
    \item DBLP database
    \end{itemize}

  \item RFC papers (technical english texts).
    
  \end{itemize}
\end{frame}

\begin{frame}
  \frametitle{Oracle vs gzip}
  
  off the shelf unix gzip
\end{frame}

\begin{frame}
  \frametitle{Oracle vs Hoffman}
  Compare text with small and large amount of entropy.
\end{frame}









% \begin{frame} 
%   \frametitle{New style -- Oracle based guidance}

%   Pros.
%      Low memory usage, only a few bits needs to be read from the oracle at a time.
%   Cons.
%      Slower checking time.
     

%   Nondeterministic checker, but with a lot of tricks it is only a small part
%   that is nondeterministic and needs the oracles help.

%   Untrusted oracles is not a problem.

  
  
% \end{frame}

\end{document}





%%% Local Variables: 
%%% mode: latex
%%% TeX-master: t
%%% End: 
