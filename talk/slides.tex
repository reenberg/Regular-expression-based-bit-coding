\documentclass[slidestop,compress,mathserif]{beamer}

\usepackage[T1]{fontenc}
\usepackage[utf8]{inputenc}
\usepackage[english]{babel} 

% Use the NAT theme in uk (also possible in DK)
\usetheme[nat,uk, footstyle=low]{Frederiksberg}

% Make overlay sweet nice by having different transparancy depending on how
% "far" ahead the overlay is. AWSOME!!
\setbeamercovered{highly dynamic}
% possible to shift back, so they are just invisible untill they should overlay
%\setbeamercovered{invisible}


% Write a short text to have that shown in the footer of slides other than the
% title slide.
\title[]{TiPL - Topics in Programming Languages}
% A possible subslide.
\subtitle{Regular-expression based bit coding}


\author[Morten Brøns-Pedersen \and Jesper Reenberg \and Nis Wegmann]
       {Morten Brøns-Pedersen \and
        Jesper Reenberg  \\
        Nis Wegmann }



% Only write DIKU in the footer of slides (except the title slide).
\institute[DIKU]{Department of Computer Science}

% Remove the date stamp from the footer of slides (except title slide) by giving
% it no short "text"
\date[]{\today} 


\begin{document}

\frame[plain]{\titlepage}

\section{Bit coding}

\subsection{Introduction}

\begin{frame}
  \frametitle{Introduction}

\end{frame}

\begin{frame}
  \frametitle{Purpose}
  
\end{frame}

\begin{frame}
  \frametitle{Advantages}
  
\end{frame}



Bitcoding
   Intro
     Formål (komprimering)
     Fordele
     xxx representation

PCC
     Short intro (ultra)
     Type check with full proof (LF) vs oracle
     Resultater (improvements) ved brug af oracle

Bitkodning - Teori (Henglein sec 5)
     Naive Usage 
     Henglein Fig 8 med praktisk eksempel på hvilke bits der kommer ud
     Pop quiz

     Figurer over oracle bits og text som ``maskiner''

Bitkodning - Praksis
     Frisch & Cardelli
     Den nøgne kode

Resultater (liv)     
     Oracle vs gzip, .. , hoffman, engelsk text
     Eksempel data: DNA, DPDA database over publikationer

\section{PCC - Proof-Carrying Code}
\begin{frame}
\frametitle{What is it?}
 WHAT

 \begin{enumerate}
 \item<1> Yay
 \item<2> Hey
 \item<3> Wush
 \item<4> Foo
 \item<5> Bar
 \item<6>{ Eksempel:
   \begin{example}
     Dette er et fint eksempel
   \end{example}}

 \end{enumerate}
\end{frame}

\begin{frame}
  \frametitle{Advantages}

   * Small burden of code consumer. Code producer must supply proof of asserted
     properties along with the code.

   * PCC is ``self-certifying''. Consumer does not need to care ...
     - of proof creation/generation.
     - of program being malicious w.r.t. assured properties.

   * PCC programs are ``tamperproof''. If edited validation will result in
     1) Proof will no longer be valid -- Rejected
     2) Proof is still valid but property is not fulfilled -- Rejected
     3) Proof is still valid and the property still holds -- Accepted

   * No trusted third party is required
     -  PCC checks the intrinsic properties and not the origin of the code.
\end{frame}

\begin{frame}
  \frametitle{Example -- Kernel module: Network Packet Filter}
  What is the right word: Module? Extension? Plugin?
\end{frame}

\begin{frame}
  \frametitle{Old style -- LF proofs}

  Pros. 
    Fast checking time.
  Cons.
    Huge memory usage. Whole proofs needs to be in memory.
  
\end{frame}

\begin{frame} 
  \frametitle{New style -- Oracle based guidance}

  Pros.
     Low memory usage, only a few bits needs to be read from the oracle at a time.
  Cons.
     Slower checking time.
     

  Nondeterministic checker, but with a lot of tricks it is only a small part
  that is nondeterministic and needs the oracles help.

  Untrusted oracles is not a problem.

  
  
\end{frame}

\end{document}





%%% Local Variables: 
%%% mode: latex
%%% TeX-master: t
%%% End: 
