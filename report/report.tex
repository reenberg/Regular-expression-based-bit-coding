\documentclass[a4paper, oneside]{memoir}
% Fixes "No room for a new \xxx" error by extending the default 256 fixed size
% LaTeX arrays
\usepackage{etex}
%\reserveinserts{28}

\usepackage[T1]{fontenc}
\usepackage[utf8]{inputenc}
\usepackage[british]{babel}

% bedre orddeling Gør at der som minimum skal blive to tegn på linien ved
% orddeling og minimum flyttes to tegn ned på næste linie. Desværre er værdien
% anvendt af babel »12«, hvilket kan give orddelingen »h-vor«.
\renewcommand{\britishhyphenmins}{22} 

% Fix of fancyref to work with memoir. Makes references look
% nice. Redefines memoir \fref and \Fref to \refer and \Refer.
% \usepackage{refer}             %
% As we dont really have any use for \fref and \Fref we just undefine what
% memoir defined them as, so fancyref can define what it wants.
\let\fref\undefined
\let\Fref\undefined
\usepackage{fancyref} % Better reference. 

\usepackage{pdflscape} % Gør landscape-environmentet tilgængeligt
\usepackage[draft]{fixme}     % Indsæt "fixme" noter i drafts.
\usepackage{hyperref}  % Indsæter links (interne og eksterne) i PDF

\usepackage{mdwtab}
\usepackage{mathenv}
\usepackage{amsfonts}
\usepackage{amsmath}
\usepackage{amssymb}
\usepackage{amsthm}
\usepackage{semantic} % for the \mathlig function

\usepackage[format=hang]{caption,subfig}
\usepackage{graphicx}
\usepackage{stmaryrd}
\usepackage[final]{listings} % Make sure we show the listing even though we are
                             % making a final report.
\usepackage{ulem} % \sout - strike-through
\usepackage{tikz}

\lstset{ %
% language=Octave,                % choose the language of the code
basicstyle=\ttfamily,        % the size of the fonts that are used for the code
basewidth=0.5em,
% numbers=left,                   % where to put the line-numbers
% numberstyle=\footnotesize,      % the size of the fonts that are used for the line-numbers
% stepnumber=2,                   % the step between two line-numbers. If it's 1 each line will be numbered
% numbersep=5pt,                  % how far the line-numbers are from the code
% backgroundcolor=\color{white},  % choose the background color. You must add \usepackage{color}
% showspaces=false,               % show spaces adding particular underscores
% showstringspaces=false,         % underline spaces within strings
% showtabs=false,                 % show tabs within strings adding particular underscores
% frame=single	                % adds a frame around the code
% tabsize=2,	                % sets default tabsize to 2 spaces
% captionpos=b,                   % sets the caption-position to bottom
% breaklines=true,                % sets automatic line breaking
% breakatwhitespace=false,        % sets if automatic breaks should only happen at whitespace
escapeinside={(@}{@)}          % if you want to add a comment within your code
}

\renewcommand{\ttdefault}{txtt} % Bedre typewriter font
%\usepackage[sc]{mathpazo}     % Palatino font
\renewcommand{\rmdefault}{ugm} % Garamond
%\usepackage[garamond]{mathdesign}

%\overfullrule=5pt
%\setsecnumdepth{part}
\setcounter{secnumdepth}{1} % Sæt overskriftsnummereringsdybde. Disable = -1.
\chapterstyle{hangnum} % changes style of chapters, to look nice.

\makeatletter
\newenvironment{nonfloatingfigure}{
  \vskip\intextsep
  \def\@captype{figure}
  }{
  \vskip\intextsep
}

\newenvironment{nonfloatingtable}{
  \vskip\intextsep
  \def\@captype{table}
  }{
  \vskip\intextsep
}
\makeatother

\renewcommand{\ttdefault}{txtt} % Bedre typewriter font
%% \usepackage[sc]{mathpazo}     % Palatino font
%% \renewcommand{\rmdefault}{ugm} % Garamond
%% \usepackage[garamond]{mathdesign}

% \overfullrule=5pt
% \setsecnumdepth{part}
\setcounter{secnumdepth}{1} % Sæt overskriftsnummereringsdybde. Disable = -1.
\chapterstyle{hangnum} % changes style of chapters, to look nice.

\theoremstyle{definition}
\newtheorem{judgment}{Judgment}
\newtheorem{definition}{Definition}
\newtheorem{lemma}{Lemma}
\newtheorem{theorem}{Theorem}
\newtheorem{corollary}{Corollary}
\newtheorem{example}{Example}

\newcommand*{\fancyrefdeflabelprefix}{def}
\fancyrefaddcaptions{english}{
  \newcommand*{\Frefdefname}{Definition}
  \newcommand*{\frefdefname}{\MakeLowercase{\Frefdefname}}
}
\frefformat{vario}{\fancyrefdeflabelprefix}{%
  \frefdefname\fancyrefdefaultspacing#1#3%
}
\Frefformat{vario}{\fancyrefdeflabelprefix}{%
  \Frefdefname\fancyrefdefaultspacing#1#3%
}

\newcommand*{\fancyreflemlabelprefix}{lem}
\fancyrefaddcaptions{english}{
  \newcommand*{\Freflemname}{Lemma}
  \newcommand*{\freflemname}{\MakeLowercase{\Freflemname}}
}
\frefformat{vario}{\fancyreflemlabelprefix}{%
  \freflemname\fancyrefdefaultspacing#1#3%
}
\Frefformat{vario}{\fancyreflemlabelprefix}{%
  \Freflemname\fancyrefdefaultspacing#1#3%
}
\frefformat{plain}{\fancyreflemlabelprefix}{%
  \freflemname\fancyrefdefaultspacing#1%
}
\Frefformat{plain}{\fancyreflemlabelprefix}{%
  \Freflemname\fancyrefdefaultspacing#1%
}

\newcommand*{\fancyrefthmlabelprefix}{thm}
\fancyrefaddcaptions{english}{
  \newcommand*{\Frefthmname}{Theorem}
  \newcommand*{\frefthmname}{\MakeLowercase{\Frefthmname}}
}
\frefformat{vario}{\fancyrefthmlabelprefix}{%
  \frefthmname\fancyrefdefaultspacing#1#3%
}
\Frefformat{vario}{\fancyrefthmlabelprefix}{%
  \Frefthmname\fancyrefdefaultspacing#1#3%
}

\newcommand*{\fancyrefcorlabelprefix}{cor}
\fancyrefaddcaptions{english}{
  \newcommand*{\Frefcorname}{Corollary}
  \newcommand*{\frefcorname}{\MakeLowercase{\Frefcorname}}
}
\frefformat{vario}{\fancyrefcorlabelprefix}{%
  \frefcorname\fancyrefdefaultspacing#1#3%
}
\Frefformat{vario}{\fancyrefcorlabelprefix}{%
  \Frefcorname\fancyrefdefaultspacing#1#3%
}

\newcommand*{\fancyrefexlabelprefix}{ex}
\fancyrefaddcaptions{english}{
  \newcommand*{\Frefexname}{Example}
  \newcommand*{\frefexname}{\MakeLowercase{\Frefexname}}
}
\frefformat{vario}{\fancyrefexlabelprefix}{%
  \frefexname\fancyrefdefaultspacing#1#3%
}
\Frefformat{vario}{\fancyrefexlabelprefix}{%
  \Frefexname\fancyrefdefaultspacing#1#3%
}
\frefformat{plain}{\fancyrefexlabelprefix}{%
  \frefexname\fancyrefdefaultspacing#1%
}
\Frefformat{plain}{\fancyrefexlabelprefix}{%
  \Frefexname\fancyrefdefaultspacing#1%
}

\newcommand{\ttt}[1]{\texttt{#1}}
\newcommand{\tnm}[1]{\textnormal{#1}}
\newcommand{\mrm}[1]{\mathrm{#1}}

\newcommand{\Cov}{\mathrm{Cov}}
\providecommand{\FV}{\mathrm{FV}}
\providecommand{\Dom}{\mathrm{Dom}}


\mathlig{||}{\parallel}
\mathlig{<'}{\prec}
\mathlig{>'}{\succ}
\mathlig{<='}{\preccurlyeq}
\mathlig{>='}{\succcurlyeq}
\mathlig{<=}{\leqslant}
\mathlig{>=}{\geqslant}
\mathlig{<>}{\neq}
\mathlig{|=}{\sqsubset}
\mathlig{=|}{\sqsupset}
\mathlig{==}{\equiv}
\mathlig{==a}{=_{\alpha}}
\mathlig{<|}{\lhd}
\mathlig{|>}{\rhd}
\mathlig{++}{\mathrel{\mbox{+\!\!\!+}}}
% ~>e or ~>g conflicts with the \cite command for some reason.
\mathlig{->e}{\stackrel{elim}{\leadsto}}
\mathlig{->g}{\stackrel{gen}{\leadsto}}

%%%%%%%%%%%%%%%%%%%%%%%%%%%%%%%%%%%%%%%%%%%%%%%%%%%%%%%%
%	    	     Forside
%%%%%%%%%%%%%%%%%%%%%%%%%%%%%%%%%%%%%%%%%%%%%%%%%%%%%%%%
\makeatletter % open mode for reading @ signed variables 
\def\maketitle{%
  \null
  \thispagestyle{empty}%
  \vfill
  \begin{center}\leavevmode
    \normalfont
    \Huge{\raggedleft \@title\par}%
    \hrulefill\par
    \Large{\raggedright \subtitle\par}%
    \vskip 2cm
    {\@date\par}%
  \end{center}%
  \vfill
\begin{minipage}{80pt}
\includegraphics*[scale=0.75]{imgs/nat-logo}
\end{minipage}
\begin{minipage}{300pt}
  \begin{flushleft}
    {\large \@author } \\
    {\footnotesize \suplementInfo }

  \end{flushleft}
\end{minipage}
\cleardoublepage % lave 1 ekstre side blank efter
  \clearpage % Terminates the page here. Everything else vil be placed on next page.
}
\makeatother % closing mode for reading @ signed variables
%%%%%%%%%%%%%%%%%%%%%%%%%%%%%%%%%%%%%%%%%%%%%%%%%%%%%%%%
%		Data til forside
%%%%%%%%%%%%%%%%%%%%%%%%%%%%%%%%%%%%%%%%%%%%%%%%%%%%%%%%
\title{Regular-expression based bit coding}
\def\subtitle{\footnotesize{TiPL - Topics in Programming Languages.}}
\author{Morten Brøns-Pedersen {\footnotesize (mortenbp@gmail.com)}\\
Jesper Reenberg {\footnotesize (jesper.reenberg@gmail.com)} \\
Nis Wegmann {\footnotesize (niswegmann@gmail.com)}}

\def\suplementInfo{
  \kern 5pt \hrule width 11pc \kern 5pt % putter 5pt spacing oven over og neden under stregen
  Dept. of Computer Science \\
  University of Copenhagen}
% \date{} % used to set explicit dates

\pagestyle{plain}

\begin{document}

\frontmatter

\maketitle
\thispagestyle{empty}


\begin{abstract}
Foo bar cat
\end{abstract}

\clearpage 

\tableofcontents*

\mainmatter

\section{Introduction}

%\cite{heni2010}

\begin{figure}
\[
\begin{array}{ccc}
  \inference{}{() : 1}
&
  \inference{}{\texttt{a} : \texttt{a}}
&
  \inference{v : E}{\mathtt{inl}\ v : E + F}
\\
\\
  \inference{v : F}{\mathtt{inr}\ v : E + F}
&
  \inference{v : E & w : F}{(v, w) : E \times F}
&
  \inference{v : 1 + E \times E^{\ast}}{\mathtt{fold}\ v : E^{\ast}}
\end{array}
\]
\caption{Inhabitation proofs for regular expressions.}
\label{fig:inhabitation_proofs}
\end{figure}

\section{Automated Derivation of Regular Expressions from XML Schemas}

\section{$\mu$-Recursion}

\subsection{Normalising $\mu$-expressions}

\section{Specializing Regexs to Strings using Huffman Trees}

The fact that the alternation operator in $Reg$ is associative and commutative with respect to language equivalence can be exploited for encoding strings more efficiently. Consider the string \texttt{"aaa"}; compressing it using the expression $E_1 = ((a + b) + c)^{*}$ yields the bit sequence $111111110$. On the other hand, compressing it using the expression $E_2 = (a + (b + c))^{*}$ yields the bit sequence $111110$. As $\mathcal{L}(E_1) = \mathcal{L}(E_2)$ both expressions can be used for compressing the excact same set of strings; for the specific string \texttt{"aaa"} we prefer $E_2$. For an arbitrary string, $s$, and an arbitrary regular expression, $E$, where $s \in \mathcal{L}(E)$, we find that the optimal method for reordering the sums in $E$, with respect to the compression ratio, is to reorder them into Huffman trees based on $s$. 

To do that, we introduce the syntactic form $Reg'_\Sigma$ over the finite alphabet $\Sigma = {a_1, \dots, a_n}$:

\[
    E ::= 0 \, | \, 1 \, | \, a \, | \, \Sigma(E_1, E_2, E_3, \dots, E_n) \, | \, E_1 \times E_2 \, | \, E^{*}
\]

The operator $\Sigma(E_1, E_2, E_3, \dots, E_n)$ denotes a sum of alternations, and the rest of the symbols denotes the same as in $Reg$. As $Reg'$ will only be used as a temporary syntactic form for reordering the sums, the order of association of the alternations inside a $\Sigma$ is insignificant.
Next we introduce what we choose to call the \emph{flattened form} of an expression $E$; to bring $E$ into the \emph{flattened form}, $E'$, we flatten out all nested sums in $E$ and rewrite them in $Reg'$ using the sum operator $\Sigma$; i.e. $(a + (b \times c + d) + e \times f)^{*} \times (c + d)$ will be rewritten into $\Sigma(a, b \times c, d, e \times f)^{*} \times \Sigma{(c, d)}$.

After having flattened out an expression $E$ into $E'$, we pick a parse tree $p_s$ given the input string $s$, such that $p_s : E$ and $||p_s|| = s$; by running trough the parse tree, we can for each sum $\Sigma(E_1, \dots, E_n)$ in $E'$ count how many times $E_k \in 1 \dots n$ is selected in the parse tree.

Finally, we can for each sum $\Sigma$ in $E'$ use these countings to reorder the $\Sigma$ back into the syntactic form of $Reg$, by ordering the alternations as Huffman trees.

\begin{example}
Consider the string $s = \mathtt{"abbc"}$ and the regular expression $E = (a + ((b + c) + d))^{*}$. Parsing $s$ using $E$ yields the following parse tree:
\[
\begin{array}{rcl}
p_s & = & \mathtt{fold}(\mathtt{inr}(\mathtt{inl} \; a, \\
    &   & \mathtt{fold}(\mathtt{inr}(\mathtt{inr}(\mathtt{inl}(\mathtt{inl} \; b), \\
    &   & \mathtt{fold}(\mathtt{inr}(\mathtt{inr}(\mathtt{inl}(\mathtt{inl} \; b), \\
    &   & \mathtt{fold}(\mathtt{inr}(\mathtt{inr}(\mathtt{inl}(\mathtt{inr} \; c), \\
    &   & \mathtt{fold}(\mathtt{inl}(()))))))))))))))
\end{array}
\]
\noindent Encoding $s$ using $e$ gives the bit sequence $101100110011010$ (15 bits). Now when translating $E$ into $Reg'$ we get the expression $E' = \Sigma{(a, b, c, d)}^{*}$. Counting the number of times each path in the sum $\Sigma{(a, b, c, d)}$ is taken in $p_s$ we get:

\begin{center}
\begin{tabular}{c|c|c|c}
$a$ & $b$ & $c$ & $d$ \\
\hline
1   & 2   & 1   & 0
\end{tabular}
\end{center}

\noindent Finally by reordering the sum into an Huffman tree using the above frequencies we get the $Reg$-expression $F = (b + (a + (c + d)))^{*}$. Encoding $s$ using $F$ gives the bit sequence $110101011100$ (12 bits); thus we have spared 3 bits.

\end{example}

\section{Implementation}

\section{Experimental Results}

\section{Conclusion}

\subsection{Future Work}

%\addcontentsline{toc}{section}{References}
%\bibliographystyle{amsplain}
%\bibliography{mybib}

%\bibliographystyle{../bibliography/theseurl}
%\bibliographystyle{amsplain}
%\bibliography{../bibliography/bibliography}

\end{document}



%%% Local Variables: 
%%% mode: latex
%%% TeX-master: t
%%% End: 
